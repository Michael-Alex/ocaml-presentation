\documentclass[titlepage,usenames,a4,landscape,semhelv]{seminar}
\usepackage{graphicx}
%\usepackage[breaklines]{listings}
\usepackage{seminar}
%\usepackage[pdftex]{color}
\usepackage{hyperref}
\input{seminar.bug}
\input{seminar.bg2}
\usepackage{tabularx}
\usepackage{fancyhdr}
\noxcomment
  \renewcommand{\slidetopmargin}{15.5mm}
  \renewcommand{\slidebottommargin}{13mm}
  %\renewcommand{\slideleftmargin}{-1cm}
  %\renewcommand{\slideleftmargin}{-1cm}
\newcommand{\bitem}[1]{\item {\bf #1}: }
  %\renewcommand{\sliderightmargin}{1cm}
\newcommand{\catfile}[1]{ \newslide {\bf #1} 
	{\scriptsize 
		\tt 
		\lstinputlisting[language=PERL,breaklines=true]{#1} 
	} 
}
\newcommand{\sslide}[1]{ \newslide
	{\huge\bf #1}
}
\newcommand{\ttt}[1]{
	``{\tt #1}''
}
\newcommand{\svg}{SoftChange SVG }
\newcommand{\php}{SoftChange PHP }
\newcommand{\gettitle}{\sf{
OCaml Tutorial
}}
\title{\gettitle}
\renewcommand{\thetitle}{\gettitle}
\newcommand{\listperl}[1]{
	{\scriptsize
		\tt
	\lstinputlisting[language=PERL,breaklines=true]{#1}
	}
}
\newcommand{\listidl}[1]{
	{\scriptsize
		\tt
	\lstinputlisting[language=IDL,breaklines=true]{#1}
	}
}
\newcommand{\hrf}[1]{
	\href{#1}{#1}
}
%\newenvironment{notes}[0]{\tt \begin{note}}{ \end{note}  }

\newcommand{\softChange}{\textsf{softChange}\ }
\newcommand{\igColumnReal}[1]{\includegraphics[width=.9\textwidth]{#1}}
\newcommand{\igH}[1]{\includegraphics[height=.9\textheight]{#1}}
\newcommand{\igReal}[1]{\includegraphics[width=.9\textwidth]{#1}}
\author{Abram Hindle\\
Edmonton Functional Programming User Group\\
http://efpug.blogspot.ca\\
abram.hindle@softwareprocess.es\\
http://softwareprocess.es/
} \date{\today}
\markboth{ }{Abram Hindle \thesection }

\newcommand{\albertaCVS}{\textsf{JReflex}\ }
\fancyhf{} % Clear all fields
\renewcommand{\headrulewidth}{1.2mm}
\renewcommand{\footrulewidth}{1.2mm}
%\fancyhead[C]{\small Abram Hindle {\bf \gettitle } CSC483A/583B}
%\fancyhead[R]{\color{white} \small Abram Hindle}
%\fancyhead[L]{\color{white}  \bf \gettitle}
%\fancyfoot[L]{\color{white} \course}
%\fancyfoot[R]{\color{white} \small \thepage}
\fancyfoot[L]{\small Abram Hindle}
\fancyhead[L]{ \bf \gettitle}
%\fancyfoot[L]{\course}
\fancyfoot[R]{\small \thepage}
\slideframe{none}
\renewcommand{\headwidth}{\textwidth}
\renewcommand{\slidetopmargin}{2cm}

%%%%%%%%%%%%%%%%%%%%%%%%%%%%%%%%%%%%%%%%%%%%%%%%%%%%%%%%%%%%%%%%%%%%%%%%%%%%%%%%
\begin{document}
\pagestyle{fancy} %bars..
%\normalcolor{white}
%\definecolor{Textcolor}{cmyk}{1,1,1,0}
\begin{slide}
\maketitle

%%%%%%%%%%%%%%%%%%%%%%%%%%%%%%%%%%%%%%%%%%%%%%%%%%%%%%%%%%%%%%%%%%%%%%%%%%%%%%%%
=OCaml
* Functional Language
* Multiple paradigms: Imperative, Functional, Object Oriented
* Heavy Generic support
* Interpreted or Byte code compiled or native
* Free as in Freedom (LGPL)
* Type Inferenced
* Cross Platform

=Why use OCaml?
* Fast, according the programming language shootouts OCaml is often better speed than even C++
* Statically Typed. Everything except marshalling is type safe. You can't break type safety without obvious hacks.
* Numerical Computation
* Performance oriented applications: statistics, mathematics, audio, multimedia
* Reasonable external library support
* Easy to integrate with existing C and C++ libraries.
* Threads (native or interpreted)

=OCaml Lists
*
\begin{verbatim}
(* construct a list *)
let l = 1 :: [] in
let l = [ 1 ; 2; 3 ] in
let l = [ 1 ] @ [ 2 ; 3 ] in
let l = 1 :: 2 :: 3 :: [] in
let fst::rest = l in
let fst::snd::third::rest = l in
\end{verbatim}

=OCaml List Operations
*
\begin{verbatim}
let third = List.nth 2 [1 ;2 ;3] in
let squares = List.map (fun x -> x * x) [ 1 ; 2 ; 3] in
let sum = List.fold_left (+) 0 [ 1 ; 2 ;  3] in
let product = List.fold_left ( * ) 1 [ 1 ; 2 ;  3] in
let gt4 = List.map ( fun x -> (x, x > 4) ) [ 1 ; 2 ; 3 ; 4 ; 5; 6 ] in
let gt4 = List.filter (fun x -> x > 4) [2 ; 4 ; 6; 8] in
let tf  = List.exists (fun x -> 10 = x) [ 1 ; 2 ; 10] in
\end{verbatim}

=OCaml Array Operations
*
\begin{verbatim}
let third = Array.nth 2 [| 1 ;2 ;3 |] in
let squares = Array.map (fun x -> x * x) [| 1 ; 2 ; 3 |] in
let sum = Array.fold_left (+) 0 [| 1 ; 2 ;  3 |] in
let product = Array.fold_left ( * ) 1 [| 1 ; 2 ;  3 |] in
let gt4 = Array.map ( fun x -> (x, x > 4) ) [| 1 ; 2 ; 3 ; 4 ; 5; 6 |] in
let gt4 = Array.filter (fun x -> x > 4) [| 2 ; 4 ; 6; 8 |] in
let tf  = Array.exists (fun x -> 10 = x) [| 1 ; 2 ; 10 |] in
\end{verbatim}

=OCaml Functions
*
\begin{verbatim}
let f x = x in
let f (a,b) = (b,a) in
let f = (* closure *)
  let x = 9 in
  (fun y -> y * x)
in
let rec f n =
  if (n > 0) then f (n - 1) else n
in
\end{verbatim}
=OCaml Functions
*
\begin{verbatim}
(* lets use pattern matching *)
let rec f = function
   0 -> 0
 | n -> f (n - 1)
in

\end{verbatim}
=OCaml Conditionals
*
\begin{verbatim}
let res = if (cond) then value1 else value2 in
let res = match x with
     x::xs -> Some (x::xs) (* pattern matching *)
   | []    -> None
in
let not_none = match x with
     None -> false
   | _ -> true
in
\end{verbatim}
=OCaml Types
*
\begin{verbatim}
let a = (x,y) ;; (* tuples can be of mixed types *)
type color = { r : int ; g : int ; b : int };;
let b = { r = 1.0 ; g = 0.5; b = 0.5 } ;; (* structs *)
type cheese = Cheese of string;; 
let c = Cheese(``Havarti'');;
type coord = int * int;;
\end{verbatim}
=OCaml types and class SML Style
*
\begin{verbatim}
type pizza =   Crust of pizza | Pepperoni | Olives 
             | Cheese of pizza list ;;
let pizza = Crust(Cheese(
   [ Pepperoni ; Olives ; Crust(Pepperoni)]
));;
let rec just_crust_and_cheese = 
  function 
     Crust(x) -> just_crust_and_cheese x
   | Cheese([]) -> true
   | Cheese(x)  -> List.for_all just_crust_and_cheese x
   | _  -> false
;;
just_crust_and_cheese (Crust(Cheese([])));;
just_crust_and_cheese pizza;;
\end{verbatim}

=OCaml line endings
* \texttt{in} means assign the value of the express to this symbol in this scope. Much like mathemtical notation
* \texttt{;} semi-colon is similar to the perl comma operator. It means ignore the return value of this expression (usually used with Unit expression)
* \texttt{;;} Used to terminated global scope, this is if you want to make globals or globally accessible functions
* \texttt{\_} Couldn't find a good slide for \texttt{\_} it just means match anything or ignore the value. Many programs are run by \texttt{let \_ = expr1 ; expr2 ; expr2 ;;}

=OCaml values are not mutable
* Most values are not mutable (arrays and strings are mutable)
* Even struct entries are not mutable. if you change them you are copying them.
        * \begin{verbatim}type foo = { num : int; mutable name: string }\end{verbatim}
* Arrays have mutable values
* References are possible: 
        * \begin{verbatim}let i = ref 0\end{verbatim}
* To change a struct or a reference:
        * \begin{verbatim}
(* deref i and add 1 to it and assign it *)
i := !i + 1; array.(!i) <- !i; (* array assn *)
(* assign a value to an entry in a struct *)
f.name <- ``lolcakes'';
\end{verbatim}

=Modules
* Modules are modular interfaces, not just a collection of types and
methods.
* Signatures define the interface:
\begin{verbatim}
module type Addable = sig 
    type t 
    val zero: t
    val add: t -> t -> t
end;;
\end{verbatim}
* Structures implement the interface:
\begin{verbatim}
module Integers = struct
    type t = int
    let zero = 0
    let add x y = x + y
end;;
\end{verbatim}
* Another client:
\begin{verbatim}
module Floats = struct
    type t = float
    let zero = 0.0
    let add x y = x +. y
end;;
\end{verbatim}

=Functors
* A big selling point of OCaml and modules is the composition of
modules via functors!
\begin{verbatim}
module AddAll = functor (X: Addable) -> struct
    type t = X.t
    let addAll x = List.fold_left X.add X.zero x
end;;
\end{verbatim}
* 
\begin{verbatim}
module IntAddAll = AddAll (Integers);;
IntAddAll.addAll( [ 1; 2; 3; 4; 5; 6 ] );;
type ilist = int list
module ILists = struct
    type t = ilist
    let zero = []
    let add ilist1 ilist2 = List.append ilist1 ilist2
end;;
module IListAdd = AddAll ( ILists );;
IListAdd.addAll([[1;2;3];[1;2;3];[1;2;3]]);;
\end{verbatim}

=Helpful OCaml modules
* The default modules handle things like Unix syscalls to do networking and some synchronization primitives. Even wimpy regexes.
* PCRE helps OCaml alot, the interface is very clear.
* Camlimages - image library
* SDL - for generaly multimedia
* Lablgtk - GTK bindings
* ocaml-gsl - Gnu Scientific Library

=OCaml Sucks 
* The comment and integer multiply cause little syntax
bugs

* Can't declare operator classes like haskell. Basically no operator
overloading. Floats and ints don't share same operator but everything
shares >, = ,< and compare

* Can't generalize classes easily (use :> operator)

* Not a lot of libraries. Not a lot of tools.

* Arrays limited to 4mb of entries. Strings are limited to 4mb in size.

* When to use ;, in, or ;; is often confusing.

* Name Spaces can clash

=OCaml Sucks pt2
* No default easy way to write binary ints or floats out to file handles or strings. 

* Some of the API is really lacking and often you need external libs to make up for it.

* Many libs are old or out of date.

* Documentation regarding the C interface is lacking (no description of how to iterate through a linked list)

* Printf is a hack. You have to declare types properly as a format not
a string to pass a template into Printf.

* Negative floating point numbers should be put in parentheses.

=OCaml debugging tips
* If you can interpret or compile to byte code you can use ocaml's interpretter to help debug
* Add more types. If you're not sure how an integer is being used stop using integers, make a type like \texttt{NumWaiters of int} to help check the types.
* If things get really painful syntactically you can always use Camlp4 but that probably won't help you debug.
* Learn how OCaml describes types, most compilation issues deal with not converting types or the compiler thinks you are using it wrong.
* When debuging start putting type hints everywhere like:
\begin{verbatim}
let fabs (x:float) = if x >= 0. then x else (-1.0) *. x in
\end{verbatim}


=OCaml summary
* Flexible language which allows for a variety programming styles
* Statically Typed
* Fast
* Sometimes cryptic and annoying
* Using OCaml's type system is like programming while writing millions of assert statements which only get run at a compile time.
* I use OCaml for performance and I use perl for text processing and web automation and general scripts.
* I didn't cover classes, modules or functors
=


\end{slide}
\end{document}
